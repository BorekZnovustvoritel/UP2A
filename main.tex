% Soubory musí být v kódování, které je nastaveno v příkazu \usepackage[...]{inputenc}

% Dokument třídy 'zpráva', vhodná pro sazbu závěrečných prací s kapitolami
% Změna typu na 'report' povolí použití i nadpisu 'chapter'
\documentclass[
    % Velikost základního písma je 12 bodů
    12pt,
    % Formát papíru je A4
    a4paper,
    % Jednostranný tisk
    oneside,
    % Záložky a metainformace ve výsledném  PDF budou v kódování unicode
    unicode,
]{article}

%%%%%%%%%%%%
% DŮLEŽITÉ %
%%%%%%%%%%%%

% Název a popis dokumentu
\title{UP2A}
\newcommand{\subtitle}{Otázky ke zkoušce}
% Autor, korektura, formátování
\author{Karma}
\newcommand{\grammar}{nikdo}
\newcommand{\formatting}{nikdo}

%%%%%%%%%%%%%%%%%%%%
% OBECNÉ NASTAVENÍ %
%%%%%%%%%%%%%%%%%%%%

% Kódování zdrojových souborů
\usepackage[utf8]{inputenc}
% Kódování výstupního souboru
\usepackage[T1]{fontenc}
% Podpora češtiny
\usepackage[czech]{babel}

% Geometrie stránky
\usepackage[
    % Vnitřní a vnější okraj
    hmargin={25mm,25mm},
    % Horní a dolní okraj
    vmargin={25mm,25mm},
    % Velikost zápatí
    footskip=17mm,
    % Vypnutí záhlaví
    nohead,
]{geometry}

% Zajištění kopírovatelnosti a prohledávanosti vytvořených PDF
\usepackage{cmap}
% Podmínky (pro použití v titulní straně)
\usepackage{ifthen}

%%%%%%%%%%%%%%%
% FORMÁTOVÁNÍ %
%%%%%%%%%%%%%%%

% Nastavení stylu nadpisů
\usepackage{sectsty}
% Formátování obsahů
\usepackage{tocloft}
% Možnost odstranění mezer mezi řádky v seznamech (pomocí '[noitemsep]')
\usepackage{enumitem}
% Sázení správných uvozovek pomocí '\enquote{}'
\usepackage{csquotes}
% Vynucení umístění poznámek pod čarou vespod stránky
\usepackage[bottom]{footmisc}

% Bezpatkové sázení nadpisů
\allsectionsfont{\sffamily}
% Změna formátování nadpisu a podnadpisů v Obsahu
\renewcommand{\cfttoctitlefont}{\Large\bfseries\sffamily}
\renewcommand{\cftsubsecdotsep}{\cftdotsep}

%%%%%%%%%%
% ODKAZY %
%%%%%%%%%%

% Tvorba hypertextových odkazů
\usepackage[
    breaklinks=true,
    hypertexnames=false,
]{hyperref}
% Nastavení barvení odkazů
\hypersetup{
    colorlinks,
    citecolor=black,
    filecolor=black,
    linkcolor=black,
    urlcolor=blue
}

%%%%%%%%%%%%%%%%%%
% OBRÁZKY, GRAFY %
%%%%%%%%%%%%%%%%%%

% Vkládání obrázků
\usepackage{graphicx}
% Nastavení popisů obrázků, výpisů a tabulek
\usepackage{caption}
% Grafy a vektorové obrázky
\usepackage{tikz}
\usetikzlibrary{shapes,arrows}

%%%%%%%%%%%%%%
% MATEMATIKA %
%%%%%%%%%%%%%%

% Sázení matematiky a matematických symbolů ('\mathbb{}')
\usepackage{amsmath}
\usepackage{amssymb}

% Náhrada za \mod a \pmod, které mají přehaně velký prostor před svou značkou
\newcommand{\Mod}[1]{\ \mathrm{mod}\ #1}
\newcommand{\Pmod}[1]{\ (\mathrm{mod}\ #1)}
% Náhrada za \forall, které kolem sebe prostor nemá žádný
\newcommand{\Forall}{\ \forall\ }

%%%%%%%%%%%%%%%%%
% ZDROJOVÉ KÓDY %
%%%%%%%%%%%%%%%%%

% Sazba zdrojových kódů
\usepackage[formats]{listings}
% Sazba zdrojových kódů
\usepackage[newfloat]{minted}
% Formátování 'minted' kódů
\usemintedstyle{pastie}
% Přepnutí prostředí 'code' do režimu výpisu kódu
\newenvironment{code}{\captionsetup{type=listing}}{}

%%%%%%%%%
% START %
%%%%%%%%%

\begin{document}

% Nastavení názvu výpisu kódu
\SetupFloatingEnvironment{listing}{name=Výpis kódu}

% Vložení titulky, obsahu a samotného textu
\makeatletter
\begin{titlepage}
    \clearpage
    \vspace*{4cm}
    \begin{center}
    \begin{minipage}{.6\textwidth}
    \begin{center}
    {\Huge \bfseries \sffamily \@title \par}
    % Pokud je '\subtitle}' prázdný text
    \ifthenelse{\equal{\subtitle}{}}
    {
        % nic nevypisuj
    }
    {
        % jinak ho vykresli velkou italikou
        \vspace{1.5cm}
        {\Large\itshape  \subtitle \par}
    }
    
    \end{center}
    \end{minipage}
    \end{center}
    % Odskočení dospod stránky
    \vfill

    % Tabulka s informacemi o dokumentu
	{
	\begin{tabular}{ ll }
        \ifthenelse{\equal{\@author}{}}{}{Text: & \@author}
        \ifthenelse{\equal{\grammar}{}}{}{\\Korektura: & \grammar}
        \ifthenelse{\equal{\formatting}{}}{}{\\Formátování: & \formatting\\}
    \end{tabular}
	}
    \hfill {\large \today\par}
    \clearpage
\end{titlepage}
\makeatother

\tableofcontents
\section{Kyberkriminalita a elektronické důkazy}

\subsection{Popište nároky kladené na zákonnost elektronického důkazu.}
Elektronický důkaz = jakýkoliv důkaz přenášen v digitální podobě. Emaily, dig. fotografie, IM historie, textové dokumenty, video a audio záznamy, provozní údaje ...

Nároky na zákonnost:\begin{itemize}
    \item Důkaz byl opatřen způsobem, který stanovuje/připouští zákon
    \item Důkaz opatřen a proveden oprávněným procesním subjektem
    \item Neoprávněné získání - absolutní(neodstranitelná)/ relativní(odstranitelná) neúčinnost
\end{itemize}
Legální evidence na základě nelegálního důkazu je nepřípustná\\
Distributivní x nedistributivní právo

\subsection{Vysvětlete strukturu §230 TZ (Neoprávněný přístup k počítačovému systému a nosiči informací). Uveďte příklady k jednotlivým odstavcům.}

\subsection{Popište rozdíly ve fungování §88 TŘ a §8 odst. 5 TŘ ve vztahu k mobilnímu telefonu}
\begin{itemize}
    \item §8 odst. 5 - povinnost součinností
    \begin{itemize}
        \item Požadujeme sdělení informací od zprostředkovatelské služby - Státní orgány...
        \item můžeme požadovat logy, metadata
        \item Toto právo lze vynucovat soudem
        \item lze také ověřit hovory a SMS v telefonu, který byl policií již zajištěn\\
    \end{itemize}
   
    \item §88 TŘ
    \begin{itemize}
        \item Může být vydán příkaz k odposlechu a záznamu telekomunikačního provozu, pokud lze důvodně předpokládat, že jím budou získány významné skutečnosti pro trestní řízení a nelze-li sledovaného účelu dosáhnout jinak nebo bylo-li by jinak jeho dosažení podstatně ztížené
    \end{itemize}
    \item Rozdíly
    \begin{itemize}
        \item §88 TŘ - menší pravděpodobnost porušení práv
        \item §8 v případě potřeby pouze metadat a přímého kontaktu s poskytovatelem
    \end{itemize}
\end{itemize}

\subsection{Vysvětlete, jaké elektronické důkazy mohou hrát roli při vyšetřování trestného činu vraždy podle §140 TZ.}

\subsection{Popište, co je to data retention a vysvětlete zásadní milníky v jeho legislativní historii (rozhodovací praxe SDEU a ÚS).}

Pojem data retention označuje ukládání provozních a lokalizačních údajů u poskytovatelů telekomunikačních služeb, převážně pro účely vyšetřování trestné činnosti.\\
Zákon č. 127/2005 Sb., o elektronických komunikacích v § 97, odst. 3
\begin{itemize}
    \item SDEU opakovaně judikoval, že plošné sledování odporuje evropskému právu i Chartě základních práv EU, 
    \item ÚS naopak došel k závěru, že česká právní úprava, která uchovávání dat přikazuje, je v pořádku
\end{itemize}







\end{document}
