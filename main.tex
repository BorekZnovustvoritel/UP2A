% Soubory musí být v kódování, které je nastaveno v příkazu \usepackage[...]{inputenc}

% Dokument třídy 'zpráva', vhodná pro sazbu závěrečných prací s kapitolami
% Změna typu na 'report' povolí použití i nadpisu 'chapter'
\documentclass[
    % Velikost základního písma je 12 bodů
    12pt,
    % Formát papíru je A4
    a4paper,
    % Jednostranný tisk
    oneside,
    % Záložky a metainformace ve výsledném  PDF budou v kódování unicode
    unicode,
]{article}

%%%%%%%%%%%%
% DŮLEŽITÉ %
%%%%%%%%%%%%

% Název a popis dokumentu
\title{UP2A}
\newcommand{\subtitle}{Otázky ke zkoušce}
% Autor, korektura, formátování
\author{Karma a Bořek}
\newcommand{\grammar}{nikdo, protože neumíme česky}
\newcommand{\formatting}{nikdo}

%%%%%%%%%%%%%%%%%%%%
% OBECNÉ NASTAVENÍ %
%%%%%%%%%%%%%%%%%%%%

% Kódování zdrojových souborů
\usepackage[utf8]{inputenc}
% Kódování výstupního souboru
\usepackage[T1]{fontenc}
% Podpora češtiny
\usepackage[czech]{babel}

% Geometrie stránky
\usepackage[
    % Vnitřní a vnější okraj
    hmargin={25mm,25mm},
    % Horní a dolní okraj
    vmargin={25mm,25mm},
    % Velikost zápatí
    footskip=17mm,
    % Vypnutí záhlaví
    nohead,
]{geometry}

% Zajištění kopírovatelnosti a prohledávanosti vytvořených PDF
\usepackage{cmap}
% Podmínky (pro použití v titulní straně)
\usepackage{ifthen}

%%%%%%%%%%%%%%%
% FORMÁTOVÁNÍ %
%%%%%%%%%%%%%%%

% Nastavení stylu nadpisů
\usepackage{sectsty}
% Formátování obsahů
\usepackage{tocloft}
% Možnost odstranění mezer mezi řádky v seznamech (pomocí '[noitemsep]')
\usepackage{enumitem}
% Sázení správných uvozovek pomocí '\enquote{}'
\usepackage{csquotes}
% Vynucení umístění poznámek pod čarou vespod stránky
\usepackage[bottom]{footmisc}

% Bezpatkové sázení nadpisů
\allsectionsfont{\sffamily}
% Změna formátování nadpisu a podnadpisů v Obsahu
\renewcommand{\cfttoctitlefont}{\Large\bfseries\sffamily}
\renewcommand{\cftsubsecdotsep}{\cftdotsep}

%%%%%%%%%%
% ODKAZY %
%%%%%%%%%%

% Tvorba hypertextových odkazů
\usepackage[
    breaklinks=true,
    hypertexnames=false,
]{hyperref}
% Nastavení barvení odkazů
\hypersetup{
    colorlinks,
    citecolor=black,
    filecolor=black,
    linkcolor=black,
    urlcolor=blue
}

%%%%%%%%%%%%%%%%%%
% OBRÁZKY, GRAFY %
%%%%%%%%%%%%%%%%%%

% Vkládání obrázků
\usepackage{graphicx}
% Nastavení popisů obrázků, výpisů a tabulek
\usepackage{caption}
% Grafy a vektorové obrázky
\usepackage{tikz}
\usetikzlibrary{shapes,arrows}

%%%%%%%%%%%%%%
% MATEMATIKA %
%%%%%%%%%%%%%%

% Sázení matematiky a matematických symbolů ('\mathbb{}')
\usepackage{amsmath}
\usepackage{amssymb}

% Náhrada za \mod a \pmod, které mají přehaně velký prostor před svou značkou
\newcommand{\Mod}[1]{\ \mathrm{mod}\ #1}
\newcommand{\Pmod}[1]{\ (\mathrm{mod}\ #1)}
% Náhrada za \forall, které kolem sebe prostor nemá žádný
\newcommand{\Forall}{\ \forall\ }

%%%%%%%%%%%%%%%%%
% ZDROJOVÉ KÓDY %
%%%%%%%%%%%%%%%%%

% Sazba zdrojových kódů
\usepackage[formats]{listings}
% Sazba zdrojových kódů
\usepackage[newfloat]{minted}
% Formátování 'minted' kódů
\usemintedstyle{pastie}
% Přepnutí prostředí 'code' do režimu výpisu kódu
\newenvironment{code}{\captionsetup{type=listing}}{}

%%%%%%%%%
% START %
%%%%%%%%%

\begin{document}

% Nastavení názvu výpisu kódu
\SetupFloatingEnvironment{listing}{name=Výpis kódu}

% Vložení titulky, obsahu a samotného textu
\makeatletter
\begin{titlepage}
    \clearpage
    \vspace*{4cm}
    \begin{center}
    \begin{minipage}{.6\textwidth}
    \begin{center}
    {\Huge \bfseries \sffamily \@title \par}
    % Pokud je '\subtitle}' prázdný text
    \ifthenelse{\equal{\subtitle}{}}
    {
        % nic nevypisuj
    }
    {
        % jinak ho vykresli velkou italikou
        \vspace{1.5cm}
        {\Large\itshape  \subtitle \par}
    }
    
    \end{center}
    \end{minipage}
    \end{center}
    % Odskočení dospod stránky
    \vfill

    % Tabulka s informacemi o dokumentu
	{
	\begin{tabular}{ ll }
        \ifthenelse{\equal{\@author}{}}{}{Text: & \@author}
        \ifthenelse{\equal{\grammar}{}}{}{\\Korektura: & \grammar}
        \ifthenelse{\equal{\formatting}{}}{}{\\Formátování: & \formatting\\}
    \end{tabular}
	}
    \hfill {\large \today\par}
    \clearpage
\end{titlepage}
\makeatother

\tableofcontents
\section{UP2A}

\subsection{Popište nároky kladené na zákonnost elektronického důkazu.}
Elektronický důkaz = jakýkoliv důkaz přenášen v digitální podobě. Emaily, dig. fotografie, IM historie, textové dokumenty, video a audio záznamy, provozní údaje ...

Nároky na zákonnost:\begin{itemize}
    \item Důkaz byl opatřen způsobem, který stanovuje/připouští zákon
    \item Důkaz opatřen a proveden oprávněným procesním subjektem
    \item Neoprávněné získání - absolutní(neodstranitelná)/ relativní(odstranitelná) neúčinnost
\end{itemize}
Legální evidence na základě nelegálního důkazu je nepřípustná\\
Distributivní x nedistributivní právo

\subsection{Vysvětlete strukturu §230 TZ (Neoprávněný přístup k počítačovému systému a nosiči informací). Uveďte příklady k jednotlivým odstavcům.}

\subsection{Popište rozdíly ve fungování §88 TŘ a §8 odst. 5 TŘ ve vztahu k mobilnímu telefonu}
\begin{itemize}
    \item §8 odst. 5 - povinnost součinností
    \begin{itemize}
        \item Požadujeme sdělení informací od zprostředkovatelské služby - Státní orgány...
        \item můžeme požadovat logy, metadata
        \item Toto právo lze vynucovat soudem
        \item lze také ověřit hovory a SMS v telefonu, který byl policií již zajištěn\\
    \end{itemize}
   
    \item §88 TŘ
    \begin{itemize}
        \item Může být vydán příkaz k odposlechu a záznamu telekomunikačního provozu, pokud lze důvodně předpokládat, že jím budou získány významné skutečnosti pro trestní řízení a nelze-li sledovaného účelu dosáhnout jinak nebo bylo-li by jinak jeho dosažení podstatně ztížené
    \end{itemize}
    \item Rozdíly
    \begin{itemize}
        \item §88 TŘ - menší pravděpodobnost porušení práv
        \item §8 v případě potřeby pouze metadat a přímého kontaktu s poskytovatelem
    \end{itemize}
\end{itemize}

\subsection{Vysvětlete, jaké elektronické důkazy mohou hrát roli při vyšetřování trestného činu vraždy podle §140 TZ.}

\subsection{Popište, co je to data retention a vysvětlete zásadní milníky v jeho legislativní historii (rozhodovací praxe SDEU a ÚS).}

Pojem data retention označuje ukládání provozních a lokalizačních údajů u poskytovatelů telekomunikačních služeb, převážně pro účely vyšetřování trestné činnosti.\\
Zákon č. 127/2005 Sb., o elektronických komunikacích v § 97, odst. 3
\begin{itemize}
    \item SDEU opakovaně judikoval, že plošné sledování odporuje evropskému právu i Chartě základních práv EU, 
    \item ÚS naopak došel k závěru, že česká právní úprava, která uchovávání dat přikazuje, je v pořádku
\end{itemize}






\subsection{Jakými předpisy je upravována oblast kybernetické bezpernosti? Čemu se věnují?}
\begin{itemize}

\item \textbf{ZoKB a NIS} - věnují se udělování práv a povinností + chránění informačních aktiv(= cokolv co je nutno chránít)
\item \textbf{Zákon č.240/200sb., krizový zákon}
    \begin{itemize}
        \item Zákon stanovuje působnost a pravomoc státních orgánů a orgánů USC a práva a povinnosti fyzických a právnických osob při přípravě na krizové situace, kterou nesouvisejí se zajišťováním obrany České republiky před vnějším napadením
    \end{itemize}
    \item \textbf{Nařízení vlády č. 432/2012 Sb.,} o kritériích pro určení prvku kritické infrastruktury
    \item \textbf{Směrnice 2016/1148} o opatřeních k zajíštění vysoké společné úrovně bezpečností sítí a informačních systémů v Unii
    \item Komu stanovujeme povinnost?
    \begin{itemize}
        \item Soukromé objekty:
        \begin{itemize}
            \item Poskytovatel služby elektronických komunikací.
            \item Osoba zajišťující váznamnou síť.
        \end{itemize}
        \item Soukromé a veřejné subjekty:
        \begin{itemize}
            \item Správce IS/KS kritické informační infrastruktury.
        \end{itemize}
        \item Veřejné subjekty:
        \begin{itemize}
            \item Správce významného informačního systému
        \end{itemize}
        \item NIS: poskytovatel základní služby/digitální služby
    \end{itemize}
\end{itemize}

\subsection{Co je to "kritická infrastruktura", "kritická informační infrastruktura", "prvek kritické infrastruktury" a 
"provozovatel prvku kritické infrastruktury"? Jak spolu tyto pojmy souvisí?}
\begin{itemize}
    \item \textbf{Kritická infrastruktura}
    \begin{itemize}
        \item většina kritické infrastruktury je v soukromých rukou, i přesto že je důležitá pro chod státu a plnění jeho úloh.
        \item S tím spojená rizika (cíl pro paralyzaci státu) i povinnosti
        \item plyn,elektřina, voda, odpad...
    \end{itemize}
    \item Kritická informační struktura:
    \begin{itemize}
        \item kybernetická složka kritické infrastruktury (ovládací systémy...)
        \item ČR zabezpečuje NÚKIB, NCKB
        \item CERT týmy (Computer Emergency Response Team)
        \begin{itemize}
            \item Národní CERT- provozuje CZ.NIC; vládní CERT
            \item NÚKIB, vládní CERT (když mám problém na kritické infrastruktuře, nahlásím to zde), Národní CERT 
(cyber je citlivá oblast, subjektům pro stát důležité anonymizuje a posílá do vládního, provozuje 
CZ.NIC
        \end{itemize}
    \end{itemize}
\end{itemize}


\subsection{ Proč současná legislativa v oblasti kybernetické bezpečnosti nestanoví povinnosti individuálním uživatelům?}
\begin{itemize}
    \item Komu stanovujeme povinnost?
    \begin{itemize}
        \item Soukromé objekty:
        \begin{itemize}
            \item Poskytovatel služby elektronických komunikací.
            \item Osoba zajišťující váznamnou síť.
        \end{itemize}
        \item Soukromé a veřejné subjekty:
        \begin{itemize}
            \item Správce IS/KS kritické informační infrastruktury.
        \end{itemize}
        \item Veřejné subjekty:
        \begin{itemize}
            \item Správce významného informačního systému
        \end{itemize}
        \item NIS: poskytovatel základní služby/digitální služby
    \end{itemize}
    \item \textbf{Jde o takové subjekty které se podílí na významné}
    \begin{itemize}
        \item Nejsou to poskytovatelé obsahu, jednotliví uživatelé, ani provozovatele jiných služeb
        \item určité subjekty jsou do činnosti zapojeny v případě tzv. kybernetického nebezpečí (pouze po určitou dobu, vyhlášeno předsedou vlády)
    \end{itemize}
\end{itemize}


\subsection{ Jaký je vztah právní úpravy kybernetické bezpečnosti a právní úpravy ochrany osobních údajů?}
\begin{itemize}
    \item Na úrovní práva:
    \begin{itemize}
        \item porušení pracovněprávní povinnosti – vedoucí zaměstnanec kontroluje, že si nikam nepíšu heslo
        \item  porušení pracovněprávní povinnosti – vedoucí zaměstnanec kontroluje, že si nikam nepíšu heslo
        \item trestněprávní odpovědnost
        \item mezinárodní odpovědnost
    \end{itemize}
    \item stát má funkce dávat:
    \begin{itemize}
        \item distributivní práva (vlastnictví, soukromí)
        \item nedistributivní práva (bezpečnost (kybernetická))
        \item hledá se rovnováha: nedistributivní práva se konstruují omezováním distributivních
    \end{itemize}
\end{itemize}

\subsection{Vysvětlete, co je to analýza rizik a jak do ní vstupují varování vydaná NÚKIB.?}
\begin{itemize}
    \item Vyhláška č. 316/2014 Sb.
    \item idk, nemůžu najít nic v přednáškách //TODO
\end{itemize}

\section{Identifikace, autentizace a datové schránky}

\subsection{Pojem elektronického podpisu a současné legislativní změny; rozdíl oproti elektronické pečeti.}
\begin{itemize}
      \item  Elektronická forma právního jednání
            \begin{enumerate}
                  \item Písemnost
                        \begin{itemize}
                              \item Elektronické prostředky nenahrazují právní jednání v písemné formě, jsou jejich jiným
                                    projevem stojícím paralelně vedle něj – jsou rovnocenné
                        \end{itemize}
                  \item Podpis jednajícího - cokoli co indetifukuje subjekt a je připojeno k dalším datům
                        \begin{itemize}
                              \item Podpis = Podpis / virtuální identita – emailová adresa, avatár, uživatelský účet, IP adresa, el.
                                    podpis, platnost od 2000, IDENTIFIKACE A INTEGRITA DOKUMENTU
                              \item Oblast el. identifikace – certifikace, ověřování, zabezpečení, spolupráce států (Amerika X Evropa)
                              \item stavěn na úroveň klasickému podpisu
                              \item Nařízení eIDAS: \uv{data v elektronické podobě, která jsou připojena k jiným datům v elektronické podobě nebo
                                          jsou s nimi logicky spojena a která podepisující osoba používá k podepsání}
                              \item Druhy: prostý, zaručený, kvalifikovaný el. podpis (certifikát)
                              \item dokument podepisuje veřejný orgán, vždy nutnost podepsat kvalifikovaným el. podpisem
                              \item  V případě podepisování dokumentu soukromou osobou v případě komunikace s veřejným
                                    orgánem – nutnost kvalifikovaného elektronického podpisu
                              \item Mimo výkon veřejné moci jakýkoliv podpis


                        \end{itemize}
            \end{enumerate}
      \item Občanský zákoník (elektronická kontraktace
      \item Nařízení Evropského parlamentu a Rady (EU) č. 910/2014 (eIDAS) ze dne 23. července 2014 o elektronické identifikaci
            a službách vytvářejících důvěru pro elektronické transakce na vnitřním trhu a o zrušení směrnice 1999/93/ES
      \item Zákon o elektronické identifikaci
      \item Zákon o elektronických úkonech a autorizované konverzi dokumentů
      \item Zákon o archivnictví a spisové službě
      \item Současné legislativní změny
      \begin{itemize}
          \item Momentálně nemáme centrálně právně ošetřené úřední ověření elektronického podpisu
          \item Od 1.2.2022 má v platnost vstoupit \textbf{Zákon o právu na digitální služby}, ten se i tímto zabývá
          \item Úřední ověření kvalifikovaným elektronickým podpisem nebo kvalifikovanou pečetí z IS veřejné správy bude mít stejnou váhu jako ověření vlastnoručním podpisem
          \item Nebo stačí uznávaný elektronický podpis + ověření podle registru obyvatel
      \end{itemize}
      \item  Elektronická pečeť slouží jako důkaz toho, že elektronický dokument vydala určitá právnická osoba, a poskytuje jistotu o původu a integritě tohoto dokumentu. \textbf{Není spojena s konkrétní osobou.}
\end{itemize}


\subsection{Nařízení eIDAS -- důvody přijetí}
\begin{itemize}
      \item eIDAS - Electronic indetification and services
      \item Nařízení Evropského parlamentu a Rady (EU) č. 910/2014 ze dne 23. července 2014 o elektronické identifikaci
            a službách vytvářejících důvěru pro elektronické transakce na vnitřním trhu a o zrušení směrnice 1999/93/ES
      \item Důvody:
            \begin{itemize}
                  \item Dotvoření digitálního volného vnitřního trhu
                  \item Zvýšení důveryhodnosti elektronických transakcí
                  \item \textbf{Vytvoření jednotného rámce pro elektronickou identifikaci}
                  \item \textbf{Sjednocení definic služeb vytvářejících důvěru}
                  \item Prokazování totožnosti v rámci EU
                  \item Nahrazení současné legislativy pro elektronické podpisy
            \end{itemize}
\end{itemize}


\subsection{Charakterizujte a popište právní otázky související s elektornickou identifikací ve smyslu prokázání totožnosti.}
\begin{itemize}
    \item Řeší to eIDAS
    \item Otázky:
    \begin{itemize}
        \item Úroveň záruky
        \begin{itemize}
            \item Nízká: jméno, heslo
            \item Značná: certifikát pro autenzizaci
            \item Vysoká: certifikát uložený na elektronickém identifikačním průkazu
        \end{itemize}
        \item Zapojí se do ní veřejný i soukromý sektor? (Podle eIDAS je soukromý optional a státy si toto stanoví)
        \item Váha elektronických a listinných dokumentů? (Stejná nebo vyšší, pokud obsahuje služby vytvářející důvěru)
        \item Služby vytvářející důvěru (el. podpis (prostý/zaručený/kvalifikovaný), pečeť, časové razítko, v eIDAS vymezeno, které to jsou), zpravidla za úplatu
        \item V ČR chybí úprava úředního ověření elektronického podpisu (řeší to zvláštní předpisy)
        \item Mezinárodní uznávání (eIDAS zaručuje, jednotná evropská identita)
        \item Archivace? (Neřeší se komplexně v eIDAS, přerazítkování)
    \end{itemize}
\end{itemize}


\subsection{Datové schránky - shrňte výhody, nevýhody a mj. se zaměřte na okamžik doručení.}
\begin{itemize}
\item Co to je ?
    \begin{itemize}
        \item Podle nařízení eIDAS čl. 3 bod 38 služba, která umožňuje přenášet data mezi třetími osobami elektronickými prostředky a poskytuje důkazy týkající se nakládání s přenášenými daty včetně dokladu o odeslání a přijetí dat, které chrání přenášená data před rizikem ztráty, krádeže, nebo poškození
        \item Spolehlivý způsob komunikace s veřejnou správou 
        \item Fikce doručení - garantuje doručení
        \item Spravováno Ministerstvem vnitra
    \end{itemize}
\item\textbf{Operace datových schránek}
    \begin{itemize}
        \item Odeslat
        \item Přijmout
        \item Ověření stavu odeslané zprávy
        \item Přijmou doklad o dodání a doručení
        \item Ověření, zda adresát má datovou schránku
    \end{itemize}
\item\textbf{Fikce doručení}
    \begin{itemize}
        \item Lhůta 10 dní - Po uplinutí této lhůty je zpráva považována za přečtenou
        \item Fakticky se jedná o povinnost pravidelně kontrolovat datovou schránku
        \item Účelem je zefektivnění procesů veřejné správy
    \end{itemize}
\item \textbf{Výhody:} el. komunikace se státními orgány, dostupnost, úspora času, nižší náklady, možnost zplnomocnění
\item \textbf{Nevýhody:} Nutnost pravidelné kontroly, omezení na 1 schránka/firma nezávisle na velikosti společnosti.
\end{itemize}


\subsection{Judikatura k elektronické identifikaci subjektu a datovým schránkám.}
\begin{itemize}
    \item Judikatura k podpisu
    \begin{itemize}
        \item II. ÚS 3042/14 - Nutnost podepsat datovou zprávu odesílanou z datové schránky elektronickým podpisem (není třeba)
        \item IV. ÚS 1829/13 – nutnost podepsat přílohu datové zprávy (není třeba)
        \item II – ÚS 3042/12 - K odeslání podání z emailu (zaručeně elektronicky podepsán) a jeho příloh (nepodepsány) – stačí jen podepsaný email
        \item II. ÚS 289/15 - Nutnost elektronicky podepsat přílohu podání z datové schránky (není nutno)
        \item KSPH 64 INS 26339/2015, 29 NSR 133/2017-B-36 (následující body)
        \begin{itemize}
            \item Nejvyšší soud zdůraznil, že úřední ověření je úzce vymezeným institutem, který je zákonem jasně definován.
            \item Soud dále poměrně vhodně rozvedl jednotlivé druhy podpisů, správně identifikoval, že zaručený elektronický podpis je jen předstupněm kvalifikovaného elektronického podpisu (rozebral poměrně kvalitně i dopady nařízení eIDAS, které upravuje problematiku elektronické identifikace)
            \item Shrnul, že požadavek úředního ověření podpisu obecně vylučuje možnost využít elektronický podpis, jelikož chybí v českém právu takové zákonné propojení. To bylo ale explicitně zakotveno v určitou dobu IZ (nebo i jiné příklady viz výše)
            \item \textbf{Ale chystá se Zákon o právu na digitální podpis, který toto bude obsahovat}
        \end{itemize}
    \end{itemize}
    \item Judikatura k datovým schránkám
    \begin{itemize}
        \item II \textbf{ÚS} 3518/11-1 – Rozhodný okamžik pro doručení datové zprávy (podání) soudu (a obecně jakémukoli veřejnému orgánu) – již doručením do příslušné datové schránky
        \item IV \textbf{ÚS} 2594/11 - Nelze se dovolávat nedoručení datové zprávy s odkazem na nedostatečné oprávnění osoby, která svým přístupem do datové schránky zapříčinila doručení datové zprávy
        \item 7 Afs (Nejvyšší správní) 60/2015-32 - Nutnost primárně doručovat do datové schránky a až poté přistoupit k alternativním způsobům doručování
        \item 21 Cdo (Nejvyšší) 5117/2014 - doručení do sekundární datové schránky v rámci komunikace mezi orgány veřejné zprávy (nenastávají všechny účinky doručení)
        \item 7 Asf 46/2010-51 – Zřízení více datových schránek (advokát, insolvenční správce, daňový poradce) – je to možné, záleží na „roli“ osoby
        \item STANOVISKO NEJVYŠŠÍHO SOUDU PLSN 1/2015 (LEDEN 2017)
        \begin{itemize}
            \item Elektronický nosič je považován za součást daného podání
            \item Dokument z oprávněné schránky, který přijde soudu, se vždy považuje za podepsaný
            \item Doručení datovou schránkou má stenou váhu jako doručení písemně s podpisem osoby
            \item DS preferovány, pokud ji má subjekt zřízenou
            \item Pokud má subjekt víc DS, vybere se ta nejvíc related
        \end{itemize}
    \end{itemize}
\end{itemize}

\section{Veřejnoprávní ochrana duševního vlastnictví}

\subsection{.Popište základní rozdíly mezi autorským právem a právy průmyslovými. Charakterizujte obecné rozdíly mezi přestupky a trestnými činy.}


\subsection{Charakterizujte a stručně popište přestupky na úseku porušování průmyslových práv. Uveďte příklad takového jednání.}


\subsection{.Charakterizujte a stručně popište přestupky na úseku autorského práva. Uveďte příklad takového jednání.}


\subsection{Charakterizujte a stručně popište trestný čin porušení chráněných průmyslových práv. Uveďte příklad takového jednání.}


\subsection{Charakterizujte a stručně popište trestný čin porušení autorského práva, práv souvisejících s právem autorským a práv k databázi. Uveďte příklad takového jednání, a to i v online prostředí}
555
\section{Informace veřejného sektoru a Otevřená data}

\subsection{Pojem \uv{informace veřejného sektoru}}
\begin{itemize}
      \item Dava vs Informace(strukturovaná data, nesou význam, zakonodárci rozdíl neberou v potaz)
      \item Veřejná správa a samospráva generují velké množství takových dat
            \begin{itemize}
                  \item Statická data
                  \item Kartografická data
                  \item Meterologické údaje
                  \item Právní info
                  \item Jízdní řády
                  \item Katastr nemovitostí
            \end{itemize}
      \item 2 druhy přístupů: právo na informace (na vyžádání), opakované využití (dostupné stále)
      \item Právo na informace je ústavní právo (Všeobecná deklarace lidských práv)
      \item Informace veřejného sektoru slouží pro kontrolu činnosti veřejného orgánu (Právo na
            informace, Zákon 106/1999 Sb. – \uv{Vystošestkovat si to}; je to i obchodní artikl)
      \item Přístup k informacím veřejného sektoru je základní politické právo
      \item Princip publicity veřejné správy – Je nutné uveřejňovat veškeré informace a přístup lze
            odepřít pouze na základě zákona (Souvisí se zásadou legality – Správní orgán jedná jen
            v mezích zákona, rozdíl oproti soukromému právu)
      \item Zákon 106/1999 Sb.:
            \begin{itemize}
                  \item Obecný předpis
                        \begin{itemize}
                              \item Speciální právní úprava pro (tady se 106 neaplikuje):
                              \item Právo na informace o životním prostředí
                              \item Katastr nemovitostí
                              \item Živnostenský zákon…
                        \end{itemize}
                  \item Implementace evropské PSI směrnice (o opakovaném použití informací veřejného
                        sektoru)
                  \item Otázky: Kdo? Jaké informace? Opravné prostředky?
            \end{itemize}
      \item Poskytování informací na žádost nebo zveřejněním
            \begin{itemize}
                  \item  Na žádost:
                        \begin{itemize}
                              \item Žádost nemá přesně formalizovanou formu, písemně, ústně (\uv{Pls, na základě
                                          zákona číslo 106/1999 chci info…})
                              \item Poplatky za poskytnutí nesmí přesáhnout náklady pro zpřístupnění
                        \end{itemize}
                  \item Zveřejněním
                        \begin{itemize}
                              \item Povinné (Kdo je povinný subjekt: paragraf 5 Zák. 106/1999)
                              \item Dobrovolné
                              \item Co nejvíc strojově zpracovatelné a znovu užitelné (strojově čitelné, otevřený
                                    formát (lze číst softwarem přístupným všem), otevřená formální norma
                                    (pravidla pro strojovou interoperabilitu))
                        \end{itemize}
            \end{itemize}
\end{itemize}


\subsection{Povinné subjekty podle zákona č. 106/1999 Sb.}
\begin{itemize}
      \item Řeší se v paragrafu 2
      \item \textbf{státní orgány, úžemní samosprávné celky a jejich orgány a veřejné instituce
      \item subjekty, kterým zákon svěřil pravomoc rozhodovat o právěch a povinnostech osob, v rozashu výkonu této pravomoci}
      \item V paragrafu 5 se pak řeší, kdo musí informace zveřejnit(subjekty, na základě zákona vedou registry, evidence, seznamy, rejstříky, které jsou na základě zákona přístupně)
      \item Veřejné instituce- řeší se v novele, cíl zajistit aplikaci na veřejnoprávní média
            \begin{itemize}
                  \item Z této novely podle rozsudku soudů vyšlo najevo, že povnné subjekty jsou i  Všeobecná zdravotní pojišťovna a fond národního majetku( řešeno ÚS)
                  \item \textbf{Státní podnik Letiště Praha} - z tohoto rozhodnutí se derivoval test pro další rozhodovací praxi
                  \item \textbf{ČEZ} - nejvyšší správní soud řekl : \uv{yup} ÚS řekl:\uv{nope, výklad je příliš široký, je to osoba soukromého práva, i když má stát většinový podíl}
            \end{itemize}
      \item Novela: \textbf{a) Státní orgán, b) ůzemní samosprávní celek, c) právnická osoba zřízená zákonem, d) právnická osoba (kde:} zřizovatel =stát \textbf{nebo} zřízená pro uspokojení veřejného zájmu \textbf{nebo} financovaná převážně státem/samosprávným celkem/právnickou osobou zřízenou zákonem), \textbf{e)veřejní podnik} poskytují informace vztahující se k jejich činnosti.
\end{itemize}


\subsection{Pojem \uv{otevřená data}}
\begin{itemize}
      \item Způsob poskytování informací veřejného prostoru
      \item Mají být
            \begin{itemize}
                  \item Ůplná
                  \item Snadno dostupná
                  \item Strojově čitelná
                  \item POužívající standardy s volně dostupnou specifikací
                  \item Zpřístupněná za jasně definovaných podmíněk užití dat s minimem omezení
                  \item Dostupná uživatelům při vynaložení minima možných nákladů
            \end{itemize}
      \item 5 stupňů otevřenosti (příklady formátů v těchto stupních: pdf/xls/json/REST API/linked data)
      \item V Londýně 500+ apps postavených na otevřených datech, investice 1 mil. Lb, návratnost 58
            mil. Lb
      \item Problém „dokopat“ ke zveřejnění, protože instituce nechtějí investovat do zveřejnění, když za
            tím nevidí vidinu zisku (nemají jistotu, že je někdo využije), programátoři nemohou stavět
            aplikace, protože neví, v jakých formátech se budou data zveřejňovat


\end{itemize}


\subsection{Obecná právní úprava otevřených dat}
V ČR
\begin{itemize}
      \item Zákon č. 106/1999 Sb., o svobodném přístupu k informací
      \item \uv{OD novela} z. č. 298/2016 Sb. – navazuje na PSI novelu
            \begin{itemize}
                  \item kvalifikovaný způsob poskytování \textbf{zveřejněním} (OD novela stošestky)
                  \item informace zveřejňované způsobem umožňujícím dálkový přístup v otevřeném
                        a strojově čitelném formátu, jejichž způsob ani účel následného využití není omezen
                        a které jsou evidovány v \textbf{národním katalogu otevřených dat} (aspoň 3 stupně otevřenosti)
            \end{itemize}
      \item \textbf{Povinná otevřená data - }údaje ze systému ARES, jízdí řády, metadata registru smluv, údaje z IS o veřejných zakázkách, dotace
      \item \textbf{Doborvolná}- pokud není zákonem vyloučeno, možno poskytnou jakékoli informace
      \item Máme národní katalog otevřených dat ( ministerstvo vnitra), metadata ke všem OD, navázanost na Evropský katalog OD
      \item Dálkový přístup k OD
      \item Data z registrů se anonymizují(bez jmen, přijmení, data narození = pouze rok... Jinak jen v OD dotací - zde je úprava speciální)
\end{itemize}
EU
\begin{itemize}
      \item PSI směrnice - zrušená, pořád implementovaná v ČR
      \item Nová open data směrnice, Česko ještě neprovedlo transpozici
            \begin{itemize}
                  \item Novinky: veřejné subjekty – nově i veřejné podniky, dynamická data, datové sady
                        s vysokou socioekonomickou hodnotou, vyloučení zvláštních práv pořizovatele
                        databáze, dopadá na údaje z výzkumu
            \end{itemize}
\end{itemize}


\subsection{Právní překážky při poskytování informací a otevírání dat}
Právo na informace x právo na ochranu osobních údajů
\begin{itemize}
      \item Základní registry se sice zveřejňují, ale je potřeba anonymizovat
      \item Platy z veřejných prostředků: v zásadě poskytovat -> protesty
      \item Řešení: neuvádět jména, jen funkce
      \item Právo odmítnout poskytnout info o platu, pokud není splněno vše z:
            \begin{itemize}
                  \item účelem vyžádání informace je přispět k diskuzi o věcech veřejného zájmu
                  \item informace samotná se týká veřejného zájmu
                  \item žadatel o informaci plní úkoly čí poslání dozo veřejnosti či roli tzv \uv{Společenského hlídacího psa}
                  \item informace existuje a je dostupný
            \end{itemize}
      \item \uv{Hlídací pes} je problematický a proti němu se nejlépe ohrazuje
      \item Pro zpracování potřeba:
            \begin{itemize}
                  \item  Zákonný účet
                  \item Právní titul (plnění zákonné povinnosti nebo oprávněný zájem správce nebo třetí
                        strany)
            \end{itemize}
      \item Anonymizace - data by němala být deanonymizovatelná
\end{itemize}
Publikace otevřených dat x duševní vlastnictví
\begin{itemize}
      \item Dála jako součást datové sady
      \item Datová sada jako originál databáze
      \item Zvláštní práva pořizovatele databáze
      \item Pokud přítomné, jsou nutné licence
            \begin{itemize}
                  \item CC-3vrstvy(pro právníky, lidi, stroje)
                  \item V OD je nejvhodnější CC BY(uvést autora)
            \end{itemize}
      \item Duševní vlastnictví se nevztahuje na prostá data
\end{itemize}

\section{Výběr z přednášek}

\subsection{Předmět a účel právní regulace elektronických komunikací}
\begin{itemize}
    \item předměty regulace:
    \begin{itemize}
        \item Zajištění sítí el. komunikací
        \item Poskytnutí služeb el. komunikací
        \item provoz přístrojů
    \end{itemize}
    \item účely:
    \begin{itemize}
        \item Zajištění funkční hospodářské soutěže, konkurence a ochrany spotřebitelů
        \item Síťová neutralita
        \item Modernizace infrastruktury
        \item Ochrana soukromí a osobních údajů (ePrivacy
    \end{itemize}
\end{itemize}


\subsection{Síťová neutralita -- popis, právní úprava v EU, aktuální výzvy v USA}
\begin{itemize}
    \item Rovné podmínky přístupu k obsahu
    \item práv. úprava EU: nařízení č. 2015/2120 - umožněn zero rating (bezplatný přístup k internetu za určitch podmínek
    \item USA : \begin{itemize}
        \item  FCC (Federal communications commission) odvolala pravidlo které zaručuje net neutrality
        \item spor Mozilla vs FCC (2/2018)
        \item 9/2018 Přijetí California Internet Consumer Protection and Net Neutrality
        \item 10/2019 - rozhodnutí federálního odvolacího soudu - FCC mohla zrušit net neutrality, ale jednotlivé státy si mohou přijmout vlastní legislativu
        \item 7/2021 Biden doporučil FCC obnovení pravidel zajišťující net neutrality
    \end{itemize}
\end{itemize}


\subsection{Základní registry -- popis, účel a právní úprava}
\begin{itemize}
    \item  Účelem je zefektivnění a využítí možností současných technologií pro online přístupy kdykoliv a odkudkolive
    \item Omezení nezbytného sběru dat a informací a jejich efektivní předávání
    \item Zákonná úprava z. č. 111/2009 Sb., o základních registrech
    \item 4 základní:
    \begin{itemize}
        \item registr obyvatel\begin{itemize}
            \item referenční údaje o fyzických osobách
            \item spravuje ministerstvo vnitra a PČR
            \item info o občanech ČR, cizincích s trvalým pobytem, cizinci, jimž byl udělejn azyl, jiné fyz. osoby u nichž zákon stanovuju že budou uvedeny v registru obyvatel
        \end{itemize} 
        \item registr osob \begin{itemize}
            \item  evidence právnických osob, organizačních složek státu, organizací s mezinárodním prvkem, podnikajících fyzických osob
            \item Obsahuje základní údaje o osobách
            \item využívají všechny orgány veřejné správy, které mají k tomuto oprávnění
        \end{itemize}
        \item registr územní identifikace\begin{itemize}
            \item Katastrální úřad, Stavební úřad, Obce
            \item info o územních prvcích, adresách, územních identifikací, územně evidenčních jednotkách
        \end{itemize}
        \item registr práv a povinností \begin{itemize}
            \item Ministerstvo vnitra
            \item Zdroj údajů pro informační systémy zákl. registrů při řízení přístupu uživatelů k údajům v jednotlivých registrech
            \item referenční údaje o právech a povinnostech osob
        \end{itemize}
    \end{itemize}
\end{itemize}

\subsection{Hlavní překážky pro zavedení komplexní eJustice v ČR}
\begin{itemize}
    \item  elektronizace soudnictví, využítí ICT v justici
    \item problémy
    \begin{itemize}
        \item Odpor justičního personálu
        \item Stereotypy, návyky
        \item Nekoncepčnost
        \item Zabezpečení
        \item Kompatibilita

    \end{itemize}
\end{itemize}

\subsection{Co je to princip distinkce a jak se vztahuje ke kyberzbraním a kyber-válce?}
\begin{itemize}
    \item Mezinárodní právo
    \item Problém dělání rozdílů mezi combatants a non-combatants
    \item V případě že pustím do sítě kyber-zbraň, je téměř nemožné zaručit že bude napdat pouze vojenské cíle, ne civilní (nemocnice atd.)
    \item Vyvstává otázka zda data jsou vůbec vojenské cíle
    \item Pokud nejsou, je útok na data porušení ženevské konvence o útoku na civilní cíle?
    
\end{itemize}

\end{document}
