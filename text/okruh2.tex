\section{Kybernetická bezpečnost}

\subsection{Jakými předpisy je upravována oblast kybernetické bezpernosti? Čemu se věnují?}
\begin{itemize}

    \item \textbf{ZoKB a NIS} - věnují se udělování práv a povinností + chránění informačních aktiv(= cokolv co je nutno chránít)
    \item \textbf{Zákon č.240/200sb., krizový zákon}
          \begin{itemize}
              \item Zákon stanovuje působnost a pravomoc státních orgánů a orgánů USC a práva a povinnosti fyzických a právnických osob při přípravě na krizové situace, kterou nesouvisejí se zajišťováním obrany České republiky před vnějším napadením
          \end{itemize}
    \item \textbf{Nařízení vlády č. 432/2012 Sb.,} o kritériích pro určení prvku kritické infrastruktury
    \item \textbf{Směrnice 2016/1148} o opatřeních k zajíštění vysoké společné úrovně bezpečností sítí a informačních systémů v Unii
    \item Komu stanovujeme povinnost?
          \begin{itemize}
              \item Soukromé objekty:
                    \begin{itemize}
                        \item Poskytovatel služby elektronických komunikací.
                        \item Osoba zajišťující váznamnou síť.
                    \end{itemize}
              \item Soukromé a veřejné subjekty:
                    \begin{itemize}
                        \item Správce IS/KS kritické informační infrastruktury.
                    \end{itemize}
              \item Veřejné subjekty:
                    \begin{itemize}
                        \item Správce významného informačního systému
                    \end{itemize}
              \item NIS: poskytovatel základní služby/digitální služby
          \end{itemize}
\end{itemize}


\subsection{Co je to "kritická infrastruktura", "kritická informační infrastruktura", "prvek kritické infrastruktury" a "provozovatel prvku kritické infrastruktury"? Jak spolu tyto pojmy souvisí?}
\begin{itemize}
    \item \textbf{Kritická infrastruktura}
          \begin{itemize}
              \item většina kritické infrastruktury je v soukromých rukou, i přesto že je důležitá pro chod státu a plnění jeho úloh.
              \item S tím spojená rizika (cíl pro paralyzaci státu) i povinnosti
              \item plyn,elektřina, voda, odpad...
          \end{itemize}
    \item Kritická informační struktura:
          \begin{itemize}
              \item kybernetická složka kritické infrastruktury (ovládací systémy...)
              \item ČR zabezpečuje NÚKIB, NCKB
              \item CERT týmy (Computer Emergency Response Team)
                    \begin{itemize}
                        \item Národní CERT- provozuje CZ.NIC; vládní CERT
                        \item NÚKIB, vládní CERT (když mám problém na kritické infrastruktuře, nahlásím to zde), Národní CERT
                              (cyber je citlivá oblast, subjektům pro stát důležité anonymizuje a posílá do vládního, provozuje
                              CZ.NIC
                    \end{itemize}
          \end{itemize}
\end{itemize}


\subsection{Proč současná legislativa v oblasti kybernetické bezpečnosti nestanoví povinnosti individuálním uživatelům?}
\begin{itemize}
    \item Komu stanovujeme povinnost?
          \begin{itemize}
              \item Soukromé objekty:
                    \begin{itemize}
                        \item Poskytovatel služby elektronických komunikací.
                        \item Osoba zajišťující váznamnou síť.
                    \end{itemize}
              \item Soukromé a veřejné subjekty:
                    \begin{itemize}
                        \item Správce IS/KS kritické informační infrastruktury.
                    \end{itemize}
              \item Veřejné subjekty:
                    \begin{itemize}
                        \item Správce významného informačního systému
                    \end{itemize}
              \item NIS: poskytovatel základní služby/digitální služby
          \end{itemize}
    \item \textbf{Jde o takové subjekty které se podílí na významné elektronické komunikaci nebo spravují systém kritické infrastruktury }
          \begin{itemize}
              \item Nejsou to poskytovatelé obsahu, jednotliví uživatelé, ani provozovatele jiných služeb
              \item určité subjekty jsou do činnosti zapojeny v případě tzv. kybernetického nebezpečí (pouze po určitou dobu, vyhlášeno předsedou vlády)
          \end{itemize}
\end{itemize}


\subsection{ Jaký je vztah právní úpravy kybernetické bezpečnosti a právní úpravy ochrany osobních údajů?}
\begin{itemize}
    \item Na úrovní práva:
          \begin{itemize}
              \item porušení pracovněprávní povinnosti – vedoucí zaměstnanec kontroluje, že si nikam nepíšu heslo
              \item  porušení pracovněprávní povinnosti – vedoucí zaměstnanec kontroluje, že si nikam nepíšu heslo
              \item trestněprávní odpovědnost
              \item mezinárodní odpovědnost
          \end{itemize}
    \item stát má funkce dávat:
          \begin{itemize}
              \item distributivní práva (vlastnictví, soukromí)
              \item nedistributivní práva (bezpečnost (kybernetická))
              \item hledá se rovnováha: nedistributivní práva se konstruují omezováním distributivních
          \end{itemize}
\end{itemize}


\subsection{Vysvětlete, co je to analýza rizik a jak do ní vstupují varování vydaná NÚKIB.}
\begin{itemize}
    \item Vyhláška č. 316/2014 Sb.
    \item Analýza rizik - určuje případné bezpečnostní riziko související s užíváním prostředků, ke kterým se vztahuje varování (př. pokud používám SQLbackend v IS, musím mít ošetřený injection, před kterým NÚKIB varoval)
    \item \uv{Máme tady něco, před čím vyšlo varování NÚKIBu?} Pokud není compliant kritická infrastruktura, můžou být postihy.
    \item V kybernetické bezpečnosti se používají performativní pravidla (cíl, ne cesta)
    \item Compliance - soulad s pravidly bezpečnosti (Pokud je někdo compliant, splňuje, co má)
    \item Používá se Risk-Based Approach - Každý si riziko musí uvědomit sám a večerku není potřeba zabezpečovat tolik jako vojenskou základnu
    \item NÚKIB vydává varování podle již vyřešených incidentů a povinným subjektům stanoví lhůtu, do které musí implementovat jím navržené zvýšení bezpečnosti
\end{itemize}
//TODO mrknout sem znova, až budeme mít prezentaci k dispozici
