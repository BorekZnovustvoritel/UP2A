\section{Kybernetická bezpečnost}

\subsection{Jakými předpisy je upravována oblast kybernetické bezpernosti? Čemu se věnují?}
\begin{itemize}

    \item \textbf{ZoKB a směrnice NIS (Network and Information Systems)} - věnují se udělování práv a povinností + chránění informačních aktiv(= cokolv co je nutno chránít)
    \item \textbf{Zákon č.240/200sb., krizový zákon}
          \begin{itemize}
              \item Zákon stanovuje působnost a pravomoc státních orgánů a orgánů USC a práva a povinnosti fyzických a právnických osob při přípravě na krizové situace, kterou nesouvisejí se zajišťováním obrany České republiky před vnějším napadením
          \end{itemize}
    \item \textbf{Nařízení vlády č. 432/2012 Sb.,} o kritériích pro určení prvku kritické infrastruktury
    \item \textbf{Směrnice 2016/1148} o opatřeních k zajíštění vysoké společné úrovně bezpečností sítí a informačních systémů v Unii
    \item Komu stanovujeme povinnost?
          \begin{itemize}
              \item Soukromé objekty:
                    \begin{itemize}
                        \item Poskytovatel služby elektronických komunikací.
                        \item Osoba zajišťující váznamnou síť.
                    \end{itemize}
              \item Soukromé a veřejné subjekty:
                    \begin{itemize}
                        \item Správce IS/KS kritické informační infrastruktury.
                    \end{itemize}
              \item Veřejné subjekty:
                    \begin{itemize}
                        \item Správce významného informačního systému
                    \end{itemize}
              \item NIS: poskytovatel základní služby/digitální služby
          \end{itemize}
\end{itemize}


\subsection{Co je to "kritická infrastruktura", "kritická informační infrastruktura", "prvek kritické infrastruktury" a "provozovatel prvku kritické infrastruktury"? Jak spolu tyto pojmy souvisí?}
\begin{itemize}
    \item \textbf{Kritická infrastruktura}
          \begin{itemize}
              \item Krizový zákon: \uv{\textit{kritickou infrastrukturou (se rozumí) prvek kritické infrastruktury nebo systém prvků kritické infrastruktury, narušení jehož funkce by mělo závažný dopad na bezpečnost státu, zabezpečení základních životních potřeb obyvatelstva, zdraví osob nebo ekonomiku státu}}
              \item Většina kritické infrastruktury je v soukromých rukou, i přesto že je důležitá pro chod státu a plnění jeho úloh.
              \item S tím spojená rizika (cíl pro paralyzaci státu) i povinnosti
              \item Plyn,elektřina, voda, odpad...
          \end{itemize}
    \item \textbf{Kritická informační struktura}
          \begin{itemize}
              \item Zákon o kybernetické bezpečnosti: \uv{\textit{kritickou informační infrastrukturou (se rozumí) prvek nebo systém prvků kritické infrastruktury v odvětví komunikační a informační systémy v oblasti kybernetické bezpečnosti}}
              \item kybernetická složka kritické infrastruktury (ovládací systémy...)
              \item ČR zabezpečuje NÚKIB, NCKB
              \item CERT týmy (Computer Emergency Response Team)
                    \begin{itemize}
                        \item Národní CERT- provozuje CZ.NIC; vládní CERT
                        \item NÚKIB, vládní CERT (když mám problém na kritické infrastruktuře, nahlásím to zde), Národní CERT
                              (cyber je citlivá oblast, subjektům pro stát důležité anonymizuje a posílá do vládního, provozuje
                              CZ.NIC
                    \end{itemize}
          \end{itemize}
          \item \textbf{Provozovatel prvku kritické infrastruktury}
          \begin{itemize}
              \item Podle krizového zákona je pojem \uv{subjekt kritické infrastruktury} definován tímto pojmem
              \item Odpovídá za ochranu prvku kritické infrastruktury
          \end{itemize}
          \item Tak trochu je vše definováno v kruzích
\end{itemize}


\subsection{Proč současná legislativa v oblasti kybernetické bezpečnosti nestanoví povinnosti individuálním uživatelům?}
\begin{itemize}
    \item Komu stanovujeme povinnost?
          \begin{itemize}
              \item Soukromé objekty:
                    \begin{itemize}
                        \item Poskytovatel služby elektronických komunikací.
                        \item Osoba zajišťující váznamnou síť.
                    \end{itemize}
              \item Soukromé a veřejné subjekty:
                    \begin{itemize}
                        \item Správce IS/KS kritické informační infrastruktury.
                    \end{itemize}
              \item Veřejné subjekty:
                    \begin{itemize}
                        \item Správce významného informačního systému
                    \end{itemize}
              \item NIS: poskytovatel základní služby/digitální služby
          \end{itemize}
    \item \textbf{Jde o takové subjekty které se podílí na významné elektronické komunikaci nebo spravují systém kritické infrastruktury }
          \begin{itemize}
              \item Nejsou to poskytovatelé obsahu, jednotliví uživatelé, ani provozovatele jiných služeb
              \item Určité subjekty jsou do činnosti zapojeny v případě tzv. kybernetického nebezpečí (pouze po určitou dobu, vyhlášeno předsedou vlády)
          \end{itemize}
    \item Pravidla jsou spíše performativní, to je v technologii poměrně žádoucí
    \item Technologie se stále vyvíjí, postupy, které se mohly zdát bezpečné kdysi, dnes mohou být prolomené
    \item Využívá se \textbf{řízení rizik} - \uv{\textit{Každému, co jeho jest}} - Nemá smysl chránit každou domácí síť správně nakonfigurovaným Kerberosem, uvažuje se o možné následné škodě
    \begin{itemize}
        \item Prevence (standard ochrany podle ISO nebo NIST, bezpečnostní politika a procedury)
        \item Reakce (řešení bezpečnostních incidentů)
        \item Vyšetření (jak tomu zabránit)
    \end{itemize}
    \item Ani nelze vymáhat po každé domácnosti, pokud se chová nezodpovědně, nelze to ani kontrolovat, protože plošné preventivní sledování je zakázáno
\end{itemize}


\subsection{ Jaký je vztah právní úpravy kybernetické bezpečnosti a právní úpravy ochrany osobních údajů?}
\begin{itemize}
    \item Na úrovní práva:
          \begin{itemize}
              \item  porušení pracovněprávní povinnosti – vedoucí zaměstnanec kontroluje, že si nikam nepíšu heslo
              \item trestněprávní odpovědnost
              \item mezinárodní odpovědnost
          \end{itemize}
    \item Stát má funkce dávat:
          \begin{itemize}
              \item distributivní práva (vlastnictví, soukromí)
              \item nedistributivní práva (bezpečnost (kybernetická))
              \item hledá se rovnováha: nedistributivní práva se konstruují omezováním distributivních
          \end{itemize}
    \item Cílem kybernetické bezpečnosti je ochrana aktiv. Co to vlastně znamená?
    \item Aktivum je v kybernetické bezpečnosti cokoliv dosažitelné přes kyberprostor
    \item Můžou to tedy dost dobře být i osobní údaje (Určitě \textit{správce významného informačního systému} bude muset splňovat všechny povinnosti správce, mezi něž patří záměrná a standardní ochrana osobních údajů a zabezpečení zpracování)
    \item Nejsou tedy mezi zákony přímé odkazy, ale fungují společně
    \item Správce osobních údajů musí posuzovat rizika
\end{itemize}


\subsection{Vysvětlete, co je to analýza rizik a jak do ní vstupují varování vydaná NÚKIB.}
\begin{itemize}
    \item Vyhláška č. 316/2014 Sb.
    \item Analýza rizik - určuje případné bezpečnostní riziko související s užíváním prostředků, ke kterým se vztahuje varování (př. pokud používám SQLbackend v IS, musím mít ošetřený injection, před kterým NÚKIB varoval)
    \item \uv{Máme tady něco, před čím vyšlo varování NÚKIBu?} Pokud není compliant kritická infrastruktura, můžou být postihy.
    \item V kybernetické bezpečnosti se používají performativní pravidla (cíl, ne cesta)
    \item Compliance - soulad s pravidly bezpečnosti (Pokud je někdo compliant, splňuje, co má)
    \item Používá se Risk-Based Approach - Každý si riziko musí uvědomit sám a večerku není potřeba zabezpečovat tolik jako vojenskou základnu
    \item Koloběh řízení rizik
    \begin{itemize}
        \item Prevence
        \begin{itemize}
            \item Standard ochrany ISO/NIST
            \item Bezpečnostní politika
            \item Bezpečnostní procedury
            \item \textbf{Hodnocení bezpečnosti}
        \end{itemize}
        \item Reakce
        \begin{itemize}
            \item Co se děje?
            \item Proč se to děje?
            \item Jak to zastavit?
            \item Řešení bezpečnostních incidentů
        \end{itemize}
        \item Vyšetření
        \begin{itemize}
            \item Jak se to stalo?
            \item Odkud to přišlo?
            \item Jak tomu zabránit?
        \end{itemize}
    \end{itemize}
    \item Pro výpočet se používá $ riziko = dopad \cdot hrozba \cdot zranitelnost $
    \item Metodiku pro přesný výpočet vydává NÚKIB
    \item NÚKIB také vydává varování podle již vyřešených incidentů a povinné subjekty \textbf{musí} tyto varování vzít v úvahu při analýze rizik
\end{itemize}
