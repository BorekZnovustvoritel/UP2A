\section{Výběr z přednášek}

\subsection{Předmět a účel právní regulace elektronických komunikací}
\begin{itemize}
    \item předměty regulace:
    \begin{itemize}
        \item Zajištění sítí el. komunikací
        \item Poskytnutí služeb el. komunikací
        \item provoz přístrojů
        \item \textbf{Služby elektronických komunikací (kromě ISP nabízející obsah)}
        \item \textbf{Síť elektronických komunikací}
        \item \textbf{\underline{NE} obsah komunikace}
        \item Zaměřuje se tedy na služby a sítě z celkových 3 vrstev e-komunikací (obsah, služby, sítě)
    \end{itemize}
    \item účely:
    \begin{itemize}
        \item Zajištění funkční \textbf{hospodářské soutěže}, \textbf{konkurence} a \textbf{ochrany spotřebitelů}
        \item Objektivita a nediskriminace, technologická neutralita, transparentnost, proporcionalita
        \item Síťová neutralita
        \item Modernizace infrastruktury
        \item Roaming
        \item Ochrana soukromí a osobních údajů (ePrivacy)
    \end{itemize}
    \item Evropský kodex pro elektronické komunikace
    \item Zákon o elektronických komunikacích - implementuje evr. směrnice a regulační rámec
    \item Nařízení vlády (např. o posuzování elektromagnetické kompatibility výrobků)
    \item Vyhlášky (např. o lokalizaci a identifikaci volajícího na tísňovou linku)
    \item Dohled - Český telekomunikační úřad, každé 3 roky analyzuje relevantní trhy a určuje nápravná opatření, pokud nefunguje hospodářská soutěž
\end{itemize}


\subsection{Síťová neutralita -- popis, právní úprava v EU, aktuální výzvy v USA}
\begin{itemize}
    \item Rovné podmínky přístupu k obsahu
    \item práv. úprava EU: nařízení č. 2015/2120 - umožněn zero rating (bezplatný přístup k internetu za určitch podmínek
    \item Zabraňuje poskytovatelům internetu zvýhodňovat nebo naopak zpomalovat/blokovat přístup na různé stránky nebo k různému obsahu
    \item USA : \begin{itemize}
        \item  FCC (Federal communications commission) odvolala pravidlo, které zaručuje net neutrality
        \item spor Mozilla vs FCC (2/2018) o zrušení rozhodnutí o zrušení síťové neutrality
        \item 9/2018 Přijetí California Internet Consumer Protection and Net Neutrality
        \item 10/2019 - rozhodnutí federálního odvolacího soudu - FCC mohla zrušit net neutrality, ale jednotlivé státy si mohou přijmout vlastní legislativu
        \item 7/2021 Biden doporučil FCC obnovení pravidel zajišťující net neutrality
    \end{itemize}
\end{itemize}


\subsection{Základní registry -- popis, účel a právní úprava}
\begin{itemize}
    \item  Účelem je zefektivnění a využítí možností současných technologií pro online přístupy kdykoliv a odkudkolive
    \item Omezení nezbytného sběru dat a informací a jejich efektivní předávání
    \item Zákonná úprava z. č. 111/2009 Sb., o základních registrech
    \item 4 základní:
    \begin{itemize}
        \item registr obyvatel\begin{itemize}
            \item referenční údaje o fyzických osobách
            \item spravuje ministerstvo vnitra a PČR
            \item info o občanech ČR, cizincích s trvalým pobytem, cizinci, jimž byl udělejn azyl, jiné fyz. osoby u nichž zákon stanovuju že budou uvedeny v registru obyvatel
        \end{itemize} 
        \item registr osob \begin{itemize}
            \item  evidence právnických osob, organizačních složek státu, organizací s mezinárodním prvkem, podnikajících fyzických osob
            \item Obsahuje základní údaje o osobách
            \item využívají všechny orgány veřejné správy, které mají k tomuto oprávnění
        \end{itemize}
        \item registr územní identifikace\begin{itemize}
            \item Katastrální úřad, Stavební úřad, Obce
            \item info o územních prvcích, adresách, územních identifikací, územně evidenčních jednotkách
        \end{itemize}
        \item registr práv a povinností \begin{itemize}
            \item Ministerstvo vnitra
            \item Zdroj údajů pro informační systémy zákl. registrů při řízení přístupu uživatelů k údajům v jednotlivých registrech
            \item referenční údaje o právech a povinnostech osob
        \end{itemize}
    \end{itemize}
\end{itemize}

\subsection{Hlavní překážky pro zavedení komplexní eJustice v ČR}
\begin{itemize}
    \item Elektronizace soudnictví, využítí ICT v justici (neplést s Online Dispute Resolution)
    \item Problémy
    \begin{itemize}
        \item Odpor justičního personálu
        \item Stereotypy, návyky
        \item Nekoncepčnost
        \item Zabezpečení
        \item Kompatibilita - elektronické spisy musí být jednotné, aby měly smysl
        \item Problém i s využitím elektronického spisu, soudy si často vedou dvojí evidenci, papírovou a elektronickou, přičemž papírová for some reason je ta primární
        \item Plně elektronický spis v našem soudnictví: centrální elektronický platební rozkaz, funguje pomocí datových schránek nebo hybridní pošty (konvertuje digitální na papírové dopisy)
        \item Neexistuje samostaný cíl eJustice, jen je to strategický rámec rozvoje veřejné správy
        \item Jedná se o zásah do soudcovské nezávislosti, protože se musí učit nové věci?
        \item Je povinné splnit podmínku rovného přístupu, ale není zaručeno, že každý disponuje elektronickým přístupem
        \item např. eISIR (elektronický insolvenční rejstřík) ztroskotal, protože ministerstvo spravedlnnosti nedokázalo zadat požadavky dostatečně tehnicky přesně
    \end{itemize}
\end{itemize}

\subsection{Co je to princip distinkce a jak se vztahuje ke kyberzbraním a kyber-válce?}
\begin{itemize}
    \item Mezinárodní právo
    \item Problém dělání rozdílů mezi combatants a non-combatants
    \item V případě že pustím do sítě kyber-zbraň, je téměř nemožné zaručit že bude napadat pouze vojenské cíle, ne civilní (nemocnice atd.)
    \item Vyvstává otázka, zda data jsou vůbec vojenské cíle
    \item Pokud nejsou, je útok na data porušení ženevské konvence o útoku na civilní cíle?
    
\end{itemize}