\section{Informace veřejného sektoru a Otevřená data}

\subsection{Pojem \uv{informace veřejného sektoru}}
\begin{itemize}
      \item Dava vs Informace(strukturovaná data, nesou význam, zakonodárci rozdíl neberou v potaz)
      \item Veřejná správa a samospráva generují velké množství takových dat
            \begin{itemize}
                  \item Statická data
                  \item Kartografická data
                  \item Meterologické údaje
                  \item Právní info
                  \item Jízdní řády
                  \item Katastr nemovitostí
            \end{itemize}
      \item 2 druhy přístupů: právo na informace (na vyžádání), opakované využití (dostupné stále)
      \item Právo na informace je ústavní právo (Všeobecná deklarace lidských práv)
      \item Informace veřejného sektoru slouží pro kontrolu činnosti veřejného orgánu (Právo na
            informace, Zákon 106/1999 Sb. – \uv{Vystošestkovat si to}; je to i obchodní artikl)
      \item Přístup k informacím veřejného sektoru je základní politické právo
      \item Princip publicity veřejné správy – Je nutné uveřejňovat veškeré informace a přístup lze
            odepřít pouze na základě zákona (Souvisí se zásadou legality – Správní orgán jedná jen
            v mezích zákona, rozdíl oproti soukromému právu)
      \item Zákon 106/1999 Sb.:
            \begin{itemize}
                  \item Obecný předpis
                        \begin{itemize}
                              \item Speciální právní úprava pro (tady se 106 neaplikuje):
                              \item Právo na informace o životním prostředí
                              \item Katastr nemovitostí
                              \item Živnostenský zákon…
                        \end{itemize}
                  \item Implementace evropské PSI směrnice (o opakovaném použití informací veřejného
                        sektoru)
                  \item Otázky: Kdo? Jaké informace? Opravné prostředky?
            \end{itemize}
      \item Poskytování informací na žádost nebo zveřejněním
            \begin{itemize}
                  \item  Na žádost:
                        \begin{itemize}
                              \item Žádost nemá přesně formalizovanou formu, písemně, ústně (\uv{Pls, na základě
                                          zákona číslo 106/1999 chci info…})
                              \item Poplatky za poskytnutí nesmí přesáhnout náklady pro zpřístupnění
                        \end{itemize}
                  \item Zveřejněním
                        \begin{itemize}
                              \item Povinné (Kdo je povinný subjekt: paragraf 5 Zák. 106/1999)
                              \item Dobrovolné
                              \item Co nejvíc strojově zpracovatelné a znovu užitelné (strojově čitelné, otevřený
                                    formát (lze číst softwarem přístupným všem), otevřená formální norma
                                    (pravidla pro strojovou interoperabilitu))
                        \end{itemize}
            \end{itemize}
\end{itemize}


\subsection{Povinné subjekty podle zákona č. 106/1999 Sb.}
\begin{itemize}
      \item Řeší se v paragrafu 2
      \item \textbf{státní orgány, územní samosprávné celky a jejich orgány a veřejné instituce
      \item subjekty, kterým zákon svěřil pravomoc rozhodovat o právech a povinnostech osob, v rozashu výkonu této pravomoci}
      \item V paragrafu 5 se pak řeší, kdo musí informace zveřejnit(subjekty na základě zákona vedou registry, evidence, seznamy, rejstříky, které jsou na základě zákona přístupné)
      \item Veřejné instituce- řeší se v novele, cíl zajistit aplikaci na veřejnoprávní média
            \begin{itemize}
                  \item Z této novely podle rozsudku soudů vyšlo najevo, že povinné subjekty jsou i  Všeobecná zdravotní pojišťovna a fond národního majetku( řešeno ÚS)
                  \item \textbf{Státní podnik Letiště Praha} - z tohoto rozhodnutí se derivoval test pro další rozhodovací praxi
                  \item \textbf{ČEZ} - nejvyšší správní soud řekl : \uv{yup} ÚS řekl:\uv{nope, výklad je příliš široký, je to osoba soukromého práva, i když má stát většinový podíl}
            \end{itemize}
      \item Novela: \textbf{a) Státní orgán, b) ůzemní samosprávní celek, c) právnická osoba zřízená zákonem, d) právnická osoba (kde:} zřizovatel =stát \textbf{nebo} zřízená pro uspokojení veřejného zájmu \textbf{nebo} financovaná převážně státem/samosprávným celkem/právnickou osobou zřízenou zákonem), \textbf{e)veřejní podnik} poskytují informace vztahující se k jejich činnosti.
\end{itemize}


\subsection{Pojem \uv{otevřená data}}
\begin{itemize}
      \item Způsob poskytování informací veřejného prostoru
      \item Mají být
            \begin{itemize}
                  \item Ůplná
                  \item Snadno dostupná
                  \item Strojově čitelná
                  \item POužívající standardy s volně dostupnou specifikací
                  \item Zpřístupněná za jasně definovaných podmíněk užití dat s minimem omezení
                  \item Dostupná uživatelům při vynaložení minima možných nákladů
            \end{itemize}
      \item 5 stupňů otevřenosti (příklady formátů v těchto stupních: pdf/xls/json/REST API/linked data)
      \item V Londýně 500+ apps postavených na otevřených datech, investice 1 mil. Lb, návratnost 58
            mil. Lb
      \item Problém „dokopat“ ke zveřejnění, protože instituce nechtějí investovat do zveřejnění, když za
            tím nevidí vidinu zisku (nemají jistotu, že je někdo využije), programátoři nemohou stavět
            aplikace, protože neví, v jakých formátech se budou data zveřejňovat


\end{itemize}


\subsection{Obecná právní úprava otevřených dat}
V ČR
\begin{itemize}
      \item Zákon č. 106/1999 Sb., o svobodném přístupu k informací
      \item \uv{OD novela} z. č. 298/2016 Sb. – navazuje na PSI novelu
            \begin{itemize}
                  \item kvalifikovaný způsob poskytování \textbf{zveřejněním} (OD novela stošestky)
                  \item informace zveřejňované způsobem umožňujícím dálkový přístup v otevřeném
                        a strojově čitelném formátu, jejichž způsob ani účel následného využití není omezen
                        a které jsou evidovány v \textbf{národním katalogu otevřených dat} (aspoň 3 stupně otevřenosti)
            \end{itemize}
      \item \textbf{Povinná otevřená data - }údaje ze systému ARES, jízdí řády, metadata registru smluv, údaje z IS o veřejných zakázkách, dotace
      \item \textbf{Doborvolná}- pokud není zákonem vyloučeno, možno poskytnou jakékoli informace
      \item Máme národní katalog otevřených dat ( ministerstvo vnitra), metadata ke všem OD, navázanost na Evropský katalog OD
      \item Dálkový přístup k OD
      \item Data z registrů se anonymizují(bez jmen, přijmení, data narození = pouze rok... Jinak jen v OD dotací - zde je úprava speciální)
\end{itemize}
EU
\begin{itemize}
      \item PSI směrnice - zrušená, pořád implementovaná v ČR
      \item Nová open data směrnice, Česko ještě neprovedlo transpozici
            \begin{itemize}
                  \item Novinky: veřejné subjekty – nově i veřejné podniky, dynamická data, datové sady
                        s vysokou socioekonomickou hodnotou, vyloučení zvláštních práv pořizovatele
                        databáze, dopadá na údaje z výzkumu
            \end{itemize}
\end{itemize}


\subsection{Právní překážky při poskytování informací a otevírání dat}
Právo na informace x právo na ochranu osobních údajů
\begin{itemize}
      \item Základní registry se sice zveřejňují, ale je potřeba anonymizovat
      \item Platy z veřejných prostředků: v zásadě poskytovat -> protesty
      \item Řešení: neuvádět jména, jen funkce
      \item Právo odmítnout poskytnout info o platu, pokud není splněno vše z:
            \begin{itemize}
                  \item účelem vyžádání informace je přispět k diskuzi o věcech veřejného zájmu
                  \item informace samotná se týká veřejného zájmu
                  \item žadatel o informaci plní úkoly čí poslání dozo veřejnosti či roli tzv \uv{Společenského hlídacího psa}
                  \item informace existuje a je dostupný
            \end{itemize}
      \item \uv{Hlídací pes} je problematický a proti němu se nejlépe ohrazuje
      \item Pro zpracování potřeba:
            \begin{itemize}
                  \item  Zákonný účet
                  \item Právní titul (plnění zákonné povinnosti nebo oprávněný zájem správce nebo třetí
                        strany)
            \end{itemize}
      \item Anonymizace - data by němala být deanonymizovatelná
\end{itemize}
Publikace otevřených dat x duševní vlastnictví
\begin{itemize}
      \item Dála jako součást datové sady
      \item Datová sada jako originál databáze
      \item Zvláštní práva pořizovatele databáze
      \item Pokud přítomné, jsou nutné licence
            \begin{itemize}
                  \item CC-3vrstvy(pro právníky, lidi, stroje)
                  \item V OD je nejvhodnější CC BY(uvést autora)
            \end{itemize}
      \item Duševní vlastnictví se nevztahuje na prostá data
\end{itemize}
