\section{UP2A}

\subsection{Popište nároky kladené na zákonnost elektronického důkazu.}
Elektronický důkaz = jakýkoliv důkaz přenášen v digitální podobě. Emaily, dig. fotografie, IM historie, textové dokumenty, video a audio záznamy, provozní údaje ...

Nároky na zákonnost:\begin{itemize}
    \item Důkaz byl opatřen způsobem, který stanovuje/připouští zákon
    \item Důkaz opatřen a proveden oprávněným procesním subjektem
    \item Neoprávněné získání - absolutní(neodstranitelná)/ relativní(odstranitelná) neúčinnost
\end{itemize}
Legální evidence na základě nelegálního důkazu je nepřípustná\\
Distributivní x nedistributivní právo

\subsection{Vysvětlete strukturu §230 TZ (Neoprávněný přístup k počítačovému systému a nosiči informací). Uveďte příklady k jednotlivým odstavcům.}

\subsection{Popište rozdíly ve fungování §88 TŘ a §8 odst. 5 TŘ ve vztahu k mobilnímu telefonu}
\begin{itemize}
    \item §8 odst. 5 - povinnost součinností
    \begin{itemize}
        \item Požadujeme sdělení informací od zprostředkovatelské služby - Státní orgány...
        \item můžeme požadovat logy, metadata
        \item Toto právo lze vynucovat soudem
        \item lze také ověřit hovory a SMS v telefonu, který byl policií již zajištěn\\
    \end{itemize}
   
    \item §88 TŘ
    \begin{itemize}
        \item Může být vydán příkaz k odposlechu a záznamu telekomunikačního provozu, pokud lze důvodně předpokládat, že jím budou získány významné skutečnosti pro trestní řízení a nelze-li sledovaného účelu dosáhnout jinak nebo bylo-li by jinak jeho dosažení podstatně ztížené
    \end{itemize}
    \item Rozdíly
    \begin{itemize}
        \item §88 TŘ - menší pravděpodobnost porušení práv
        \item §8 v případě potřeby pouze metadat a přímého kontaktu s poskytovatelem
    \end{itemize}
\end{itemize}

\subsection{Vysvětlete, jaké elektronické důkazy mohou hrát roli při vyšetřování trestného činu vraždy podle §140 TZ.}

\subsection{Popište, co je to data retention a vysvětlete zásadní milníky v jeho legislativní historii (rozhodovací praxe SDEU a ÚS).}

Pojem data retention označuje ukládání provozních a lokalizačních údajů u poskytovatelů telekomunikačních služeb, převážně pro účely vyšetřování trestné činnosti.\\
Zákon č. 127/2005 Sb., o elektronických komunikacích v § 97, odst. 3
\begin{itemize}
    \item SDEU opakovaně judikoval, že plošné sledování odporuje evropskému právu i Chartě základních práv EU, 
    \item ÚS naopak došel k závěru, že česká právní úprava, která uchovávání dat přikazuje, je v pořádku
\end{itemize}





