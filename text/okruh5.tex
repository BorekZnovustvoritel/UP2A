\section{Veřejnoprávní ochrana duševního vlastnictví}

\begin{itemize}
    \item\textbf{Před zahájením sporu:} 
    \begin{itemize}
        \item Zjišťování informací
        \item Předžalobní výzva – 7 dnů před podáním návrhu na zahájení řízení; právo na náhradu nákladů řízení proti
        žalovanému
        \item Žaloba (jen v soukromém právu) – nejasný rozsah (spory o nárocích výcházejících z průmyslového vlastnictví)
    \end{itemize}
    \item\textbf{Během sporu:}
    \begin{itemize}
        \item Předběžná opatření – tyhle věci přemístíme sem, aby se např. s důkazy nic nestalo (soud příslušný k řízení)
    \end{itemize}
    \item\textbf{Vymáhání práv dle AZ:}
    \begin{itemize}
        \item Ohrožovací delikt
        \item Nároky – určení autorství, zákaz ohrožení (hrozícího opakování), informace, reparace (odstranění následků),
        satisfakce, zákaz poskytování služby
        \item Uveřejnění rozsudku, omluva – pozor na znění smlouvy (např. na rozsah omluvy)
        \item Nutná přiměřenost závažnosti porušení práva, nutno zohlednit zájmy zúčastněných osob
        \item Specifický způsob stanovení výše škody a BO
        \begin{itemize}
            \item ušlý zisk x škoda scházející
            \item místo ušlého zisku možnost 1x ceny licence (netřeba prokazovat výši ušlého zisku)
            \item u BO 2x ceny licence
            \item v rámci satisfakce zejména omluva => pokud nedostatečné, teprve peníze
            \item přihlíží se k závažnosti újmy, k okolnostem případu…
        \end{itemize}
        \item DRM
    \end{itemize}
    \item\textbf{Vymáhání práv dle ZVPPZ}
    \begin{itemize}
        \item Právo na informace
        \item Opatření k nápravě
        \begin{itemize}
            \item Zdržovací (ohrožovací delikt)
            \item Odstraňovací (stažení výrobku z trhu)
            \item Proporcionalita práv třetích osob
            \item Zveřejnění rozsudku
            \item Namísto opatření možno nařídit zaplacení peněžního vyrovnání
        \end{itemize}
        \item Specifický způsob stanovení výše škody a BO
        \begin{itemize}
            \item pokud věděl, že porušuje -> nejméně 2x ceny licence
            \item pokud nevěděl, ani nemohl -> nejméně 1x ceny licence
        \end{itemize}
        \item přiměřené zadostiučinění i peněžité; zohlednění všech okolností
    \end{itemize}
    
\end{itemize}

\subsection{Popište základní rozdíly mezi autorským právem a právy průmyslovými. Charakterizujte obecné rozdíly mezi přestupky a trestnými činy.}
//TODO

\subsection{Charakterizujte a stručně popište přestupky na úseku porušování průmyslových práv. Uveďte příklad takového jednání.}
Přestupek (dle 5§ ZOP) = společensky škodlivý protiprávní čin, který je v zákoně za přestupek výslovně označen
a který vykazuje znaky stanovené zákonem, nejde-li o trestný čin
\underline{přestupek mohou udělit:}
\begin{itemize}
    \item Úřady obcí s rozšířenou působností
    \item Dozorové orgány u ochrany spotřebitele (čeká obchodní inspekce, státní zemědělská a potrav.
    \item Inspekce, ústav pro kontrolu léčiv, veterinární správa)
    \item Celní správa + unijní legislativa (bezpečnostní sbor)
\end{itemize}
Průmyslové právo patří do práva duševního vlastnictví, stejně jako autorské právo (tedy ochrana nehmotných
statků).
\underline{Patří sem:}
\begin{itemize}
    \item Patent - zákon č. 527/1990 Sb., o vynálezech…
    \item Ochranná známka (441/2003)
    \item Průmyslový vzor (207/2000)
    \item Chráněné zeměpisné označení, o značení původu (452/2001)
    \item Obchodní firma, obchodní tajemství, know-how, dom=nová práva, odrůdy, plemena, užité vzory, …
\end{itemize}
O sporech o průmyslové vlastnictví/práva rozhoduje \underline{Městský soud v Praze} (§6 ZVPPV)
\textbf{Oprávněné osoby:}
\begin{itemize}
    \item Majitel/vlastník dle dílčího předpisu
    \item Profesní organizace ochrany práv uznaná k zastupování majitele/vlastníka
    \item Nabyvatel licence (se souhlase majitele/vlastníka) - 1 měsíc presumpce
\end{itemize}
\textbf{Přestupky:}
\begin{itemize}
    \item Živelný institut (aktuální úprava od 1.7.2017; skutkové podstaty dle dílčích právních předpisů + ZNP §8
    + § 10)
    \item Fyzické osoby (zavinění ve formě nedbalosti, není-li vyžadován úmysl (§15))
    \item Právnické osoby/podnikající osoby (objektivní odpovědnost, možnost zproštění se při prokázání úsilí
    (§21)
    \item Vybrané skutkové podstaty dle ZNP (přestupky proti majetku - §8; přestupky na úseku porušování práv
    k obchodní firmě (§10)
    \item Přestupkové řízení dle ZOP
    \item Správní odpovědnost -> správní trest
    \begin{itemize}
        \item Napomenutí, pokuta, zákaz činnosti, propadnutí věci nebo náhradní hodnoty, zveřejnění
rozhodnutí o přestupku, proporcionalita a kombinace těchto
    \end{itemize}
\end{itemize}
\textbf{Příklad takového jednání: Pirátská strana vs LEGO}
\begin{itemize}
    \item pirátska strana uverejnila na yt politický spot, ktorý obsahoval panáčikov
    LEGO – spoločnosť LEGO poslala pirátskej strane list, že porušili ochrannú
    známku, zasahujú do dobrého mena spoločnosti a nekalej súťaže
    \item  LEGO si neprialo aby ich figúrky boli spájané s politickou stranou,
    respektíve aby si ľudia nemysleli, že LEGO podporuje danú politickú stranu
    a súhlasí s ich názormi
    \item  LEGO sa obrátilo ma mestský súd v Prahe aby vydal predbežné opatrenie
    aby bolo video stiahnuté – súd vyhovel – piráti museli video stiahnuť – dali
    ho ako súkromné
    \item  LEGO podalo žalobu, súd vyhovel – pirátska strana sa musela zdržať
    používania ochranných známok LEGO a uverejniť ospravedlnenie
    \item  piráti sa obhajovali tým, že LEGO vystavuje svojich panáčikov prakticky
    všade verejne a teda „umelci“ ich môžu naďalej používať, ale všetky
    odvolania dopadli neúspešne, momentálne je vec na ústavnom súde

\end{itemize}

\subsection{Charakterizujte a stručně popište přestupky na úseku autorského práva. Uveďte příklad takového jednání.}


\subsection{Charakterizujte a stručně popište trestný čin porušení chráněných průmyslových práv. Uveďte příklad takového jednání.}


\subsection{Charakterizujte a stručně popište trestný čin porušení autorského práva, práv souvisejících s právem autorským a práv k databázi. Uveďte příklad takového jednání, a to i v online prostředí}
