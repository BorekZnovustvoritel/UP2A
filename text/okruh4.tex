
\section{Veřejnoprávní ochrana duševního vlastnictví}



\subsection{Popište základní rozdíly mezi autorským právem a právy průmyslovými. Charakterizujte obecné rozdíly mezi přestupky a trestnými činy.}
\begin{itemize}
    \item \textbf{Autorské právo}\begin{itemize}
        \item Duševní dílo
        \item Dle AZ
        \item Řeší krajské soudy dle bydliště obžalovaného
        \item Oprávněné osoby:\begin{itemize}
            \item Autor(dědic)
            \item vykonavatel práv (omezeně)
            \item Nabyvatel váhradní licence (omezeně)
        \end{itemize}
    \end{itemize}
    \item \textbf{Průmyslová práva}\begin{itemize}
        \item dle ZVPPV
        \item Městský soud  v Praze
        \item Oprávněné osoby:\begin{itemize}
            \item Majitel 
            \item Profesní organizace ochrany práv uznaná k zastoupení majitele
            \item nabyvatel licence (souhlas majitele) - 1 měsíc presumpce souhlasu pokud majitel nespolupracuje
        \end{itemize}
    \end{itemize}
    \item \textbf{Přestupek:}\begin{itemize}
        \item Společensky škodlivý protiprávní čin, který je v zákoně vyslověně označen přestupkem a vykazuje znaky stanovené zákonem, nejde-li o trestný čin
        \item fyzická i právnická osoba
    \end{itemize}
    \item \textbf{Trestný čin}\begin{itemize}
        \item Protiprávní čin, který trestní zákon označuje za trestný a který vykazuje znaky uvedené v takovém zákoně
    \end{itemize}
    
\end{itemize}

\subsection{Charakterizujte a stručně popište přestupky na úseku porušování průmyslových práv. Uveďte příklad takového jednání.}
Přestupek (dle 5§ ZOP) = společensky škodlivý protiprávní čin, který je v zákoně za přestupek výslovně označen
a který vykazuje znaky stanovené zákonem, nejde-li o trestný čin
\newline\newline\underline{Přestupek mohou udělit:}
\begin{itemize}
    \item Úřady obcí s rozšířenou působností
    \item Dozorové orgány u ochrany spotřebitele (čeká obchodní inspekce, státní zemědělská a potrav.
    \item Inspekce, ústav pro kontrolu léčiv, veterinární správa)
    \item Celní správa + unijní legislativa (bezpečnostní sbor)
\end{itemize}
Průmyslové právo patří do práva duševního vlastnictví, stejně jako autorské právo (tedy ochrana nehmotných
statků).
\newline\newline\underline{Patří sem:}
\begin{itemize}
    \item Patent - zákon č. 527/1990 Sb., o vynálezech…
    \item Ochranná známka (441/2003)
    \item Průmyslový vzor (207/2000)
    \item Chráněné zeměpisné označení, o značení původu (452/2001)
    \item Obchodní firma, obchodní tajemství, know-how, dom=nová práva, odrůdy, plemena, užité vzory, …
\end{itemize}
O sporech o průmyslové vlastnictví/práva rozhoduje \underline{Městský soud v Praze} (§6 ZVPPV)
\newline\newline\textbf{Oprávněné osoby:}
\begin{itemize}
    \item Majitel/vlastník dle dílčího předpisu
    \item Profesní organizace ochrany práv uznaná k zastupování majitele/vlastníka
    \item Nabyvatel licence (se souhlase majitele/vlastníka) - 1 měsíc presumpce
\end{itemize}
\textbf{Přestupky:}
\begin{itemize}
    \item Živelný institut (aktuální úprava od 1.7.2017; skutkové podstaty dle dílčích právních předpisů + ZNP §8
    + § 10)
    \item Fyzické osoby (zavinění ve formě nedbalosti, není-li vyžadován úmysl (§15))
    \item Právnické osoby/podnikající osoby (objektivní odpovědnost, možnost zproštění se při prokázání úsilí
    (§21)
    \item Vybrané skutkové podstaty dle ZNP (přestupky proti majetku - §8; přestupky na úseku porušování práv
    k obchodní firmě (§10)
    \item Přestupkové řízení dle ZOP
    \item Správní odpovědnost -> správní trest
    \begin{itemize}
        \item Napomenutí, pokuta, zákaz činnosti, propadnutí věci nebo náhradní hodnoty, zveřejnění
rozhodnutí o přestupku, proporcionalita a kombinace těchto
    \end{itemize}
\end{itemize}
\textbf{Příklad takového jednání: Pirátská strana vs LEGO}
\begin{itemize}
    \item pirátska strana uverejnila na yt politický spot, ktorý obsahoval panáčikov
    LEGO – spoločnosť LEGO poslala pirátskej strane list, že porušili ochrannú
    známku, zasahujú do dobrého mena spoločnosti a nekalej súťaže
    \item  LEGO si neprialo aby ich figúrky boli spájané s politickou stranou,
    respektíve aby si ľudia nemysleli, že LEGO podporuje danú politickú stranu
    a súhlasí s ich názormi
    \item  LEGO sa obrátilo ma mestský súd v Prahe aby vydal predbežné opatrenie
    aby bolo video stiahnuté – súd vyhovel – piráti museli video stiahnuť – dali
    ho ako súkromné
    \item  LEGO podalo žalobu, súd vyhovel – pirátska strana sa musela zdržať
    používania ochranných známok LEGO a uverejniť ospravedlnenie
    \item  piráti sa obhajovali tým, že LEGO vystavuje svojich panáčikov prakticky
    všade verejne a teda „umelci“ ich môžu naďalej používať, ale všetky
    odvolania dopadli neúspešne, momentálne je vec na ústavnom súde

\end{itemize}

\subsection{Charakterizujte a stručně popište přestupky na úseku autorského práva. Uveďte příklad takového jednání.}
Opět jde o ochranu nehmotných statků, \underline{přestupek mohou udělit:}
\begin{itemize}
    \item Úřady obcí s rozšířenou působností
    \item Dozorové orgány u ochrany spotřebitele (čeká obchodní inspekce, státní zemědělská a potrav.
    Inspekce, ústav pro kontrolu léčiv, veterinární správa)
    \item Celní správa + unijní legislativa (bezpečnostní sbor)
\end{itemize}

Přestupek definován v předchozí kapitole\newline
\newline\underline{Správní delikt: }ozančení pro všechny veřejnoprávní delikty, které zpracovává správní úřad; delikt - porušení práva a nebo s ním stanovené povinnosti.
\newline
\newline\textbf{Přestupek/právní delikt podle §105a a §105b AZ}
\begin{itemize}
    \item osoba se dopustí přestupku tím, že:
    \begin{itemize}
        \item neoprávněně použije autorské dílo, umělecký výkon, zvukový či
         zvukově-obrazový záznam, rozhlasové nebo televizní vysílání...
         (pokuta až 150 000 Kč)
        \item neoprávněne zasahuje do autorského práva (pokuta 100 000 Kč)
        \item jako obchodník, který se účastní prodeje originálů uměleckého
         díla, nesplní oznamovací povinnost (pokuta 50 000 Kč)
    \end{itemize}
\end{itemize}
\textbf{Příklady:} výroba a rozšiřování kopií DVD filmů (pro vlastní potřebu je to ok),
sdílení přes torrent, neoprávněné veřejné promítání filmů – v autobuse,
letadle, klubu – bez oprávnění

\subsection{Charakterizujte a stručně popište trestný čin porušení chráněných průmyslových práv. Uveďte příklad takového jednání.}

\begin{itemize}
    \item \textbf{Trestný čin (všeobecně)} – protiprávní čin, dělí se na přečin (trestní čin spáchaný
    z nedbalosti) a zločin (úmyslný trestný čin) – pachatelem může být pouze fyzická
    osoba
    \item trestný čin podle §269 (40/2009 - trestní zákoník) - porušení průmyslových
    práv
    \begin{itemize}
        \item kdo neoprávněně zasáhne i když nepatrně do práv k chráněnému
        vynálezu, průmyslovému vzoru nebo topografii polovodičového výrobku,
        bude potrestán odnětím svobody až na 2 roky, zákazem činnosti nebo
        propadnutím věci
        \item odnětím svobody na 6 měsíců až 5 let, peněžitým trestem nebo
        propadnutím věci bude pachatel potrestán, jestliže:
        \begin{itemize}
            \item čin vykazuje znaky obchodní činnosti nebo jiného podnikání
            \item získá takovým činem pro sebe nebo někoho značný prospěch
            \item dopustí se takového činu ve značném rozsahu
        \end{itemize}
        \item odnětí svobody na 3-8 let pokud:
        \begin{itemize}
            \item získá činem pro sebe nebo někoho jiného prospěch ve velkém
            rozsahu
            \item dopustí se činu ve velkém rozsahu
        \end{itemize}
    \end{itemize}
    \item \textbf{Příklady:} okopírování iPhone a následný prodej, všeobecné zneužití
    průmyslového vzoru (design, tvar, materiál, obrys, struktura), neoprávněné použití
    patentu někoho jiného pro vlastní obchodní účely
\end{itemize}



\subsection{Charakterizujte a stručně popište trestný čin porušení autorského práva, práv souvisejících s právem autorským a práv k databázi. Uveďte příklad takového jednání, a to i v online prostředí}
\begin{itemize}
    \item Zde může být pachatelem fyzická i právnická osoba
    \item §270 (40/2009 - trestní zákoník) - v podstatě stejný zákon jako pro
    průmyslová práva, jen se záměnou slovíčka vynález za autorské právo:
    \begin{itemize}
        \item kdo neoprávněně zasáhne i když nepatrně do práv k chráněnému
        autorskému dílu, průmyslovému vzoru nebo topografii polovodičového
        výrobku, bude potrestán odnětím svobody až na 2 roky, zákazem činnosti
        nebo propadnutím věci
        \item odnětím svobody na 6 měsíců až 5 let, peněžitým trestem nebo
        propadnutím věci bude pachatel potrestán, jestliže:
        \begin{itemize}
            \item čin vykazuje znaky obchodní činnosti nebo jiného podnikání
            \item získá takovým činem pro sebe nebo někoho značný prospěch
            \item dopustí se takového činu ve značném rozsahu
        \end{itemize}
        \item odnětí svobody na 3-8 let pokud:
        \begin{itemize}
            \item získá činem pro sebe nebo někoho jiného prospěch ve velkém
            rozsahu
            \item dopustí se činu ve velkém rozsahu
        \end{itemize}
    \end{itemize}
    \item \textbf{Příklady:} prodej nelegálních kopií filmů na DVD (velké zisky), zveřejnění ještě
    nevysílaného filmu, seriálu, celkově obchodování s nelegálně získanými
    filmy, hudbou a jejich rozšiřování
\end{itemize}
